\newcommand{\HT}[1]{\mbox{$ht[#1]$}}
\newcommand{\pczero}{\mbox{\texttt{init}}}
%\newcommand{\pcone}{\mbox{ready\hspace{-0.5mm}-to\hspace{-0.5mm}-add}}
\newcommand{\pcone}{\mbox{\texttt{ready-to-add}}}
\newcommand{\pctwo}{\mbox{\texttt{added}}}
\newcommand{\pcthree}{\mbox{\texttt{error}}}
\newcommand{\negate}[1]{\mbox{$\neg\ #1$}}
\newcommand{\incr}{\xspace\mbox{$+\hspace{-1mm}=$}\xspace}
\newcommand{\decr}{\xspace\mbox{$-\hspace{-1mm}=$}\xspace}

\section{Verification}
\label{sec:veri}

We have modeled a simplified version of the \texttt{lfht\_add}
operation and proved some facts about the growing and data migration
mechanism.

    
\begin{figure}[t]
\begin{tikzpicture}[shorten >=1pt, node distance=5cm, on grid]
  \node[draw,rectangle]   (q_0)      {$\pczero$};
  \node[draw,rectangle] (q_1) [right=of q_0] {$\pcone$};
  \node[draw,rectangle] (q_2) [right=of q_1] {$\pctwo$};
  \node[draw,rectangle] (q_3) [right=of q_2] {$\pcthree$};
  
  \path[->](q_0) edge [bend left=45] node [below] {\textbf{migrate}} (q_1);
  \path[->](q_0) edge [bend right]   node [below] {\textbf{no-migrate}} (q_1);
  \path[->](q_1) edge                node [below] {\textbf{full}} (q_0);
  \path[->](q_1) edge                node [above] {\textbf{add}} (q_2);
  \path[->](q_2) edge [bend right=50]  node [above] {\textbf{grow}} (q_0);
  \path[->](q_2) edge [bend left=50]  node [above] {\textbf{no-grow}} (q_0);  
  \path[->](q_2) edge                node [above] {\textbf{out-of-mem}} (q_3);  
  \path[->](q_3) edge [loop above]   node  {} ();
\end{tikzpicture}
\caption{Configuration diagram for the \texttt{lfht\_add} operation}
\label{fig:model}
\end{figure}



\subsection{Abstract Model for \texttt{lfht\_add}}

The \texttt{lfht\_add} operation has been modeled as a state machine
by specifying a transition relation on a set of state variables. This
transition relation is stated as guarded commands, where the guard
triggers a particular transition.  We first describe the state
variables and then all the machine transitions.

\subsubsection{State Variables}

A variable \texttt{state} records one of the four possible states
shown in the configuration diagram in Figure~\ref{fig:model}:
\emph{\pczero}, \emph{\pcone}, \emph{\pctwo}, and \emph{\pcthree}.
%
From \pczero, the initial state, the machine can transition to
\pcone\ by taking two different actions: migrating some entries from
the old table to the new one (\textbf{migrate}) or as a non-op action
in case that no migration is required (\textbf{no-migrate}). Only when
the machine is at \pcone, the entry can be added into the new
table. The state machine can either move to \pctwo\ if there is space
in the new table (\textbf{add}) or it can go back to the initial state
if the new table is full (\textbf{full}). Once a new entry has been
added into the new table, the hash table might need to grow. If this
is the case, then the state machine goes back to \pczero\ by taking
the transition \textbf{grow}, otherwise it goes also to \pczero\ but
taking this time the transition \textbf{no-grow}. If there is no more
available memory to allocate then the state machine moves to an error
state indicating that the there is no more memory and the machine
stays there forever.

The other key state variable is \texttt{ht} which represents the hash
table in our abstract model.  Since we allow the hash table to grow by
making a new copy and the old copy cannot be immediately freed due to
thread interactions, our model must keep track of both the new table
and all its old copies. Taking this into consideration, the hash table
is modeled by an array of size $N$ of \texttt{header} elements. The
array is indexed starting from $1$. The type \texttt{header} is a
record of \emph{num\_entries}, \emph{num\_to\_migrate}, and
\emph{assimilated} where \emph{num\_entries} is a non-negative integer
that keeps track of the number of entries in the table,
\emph{num\_to\_migrate} is another non-negative integer that
represents the number of entries that needs to be migrated to the new
table, and \emph{assimilated} is a boolean that indicates whether the
table copy has been assimilated or not. Note that our abstract model
ignores the contents of the actual hash table since they are
irrelevant to the properties we are interested. \todo{list limitations
  of the model (e.g., counters updates are considered atomic, no
  handle)}. For convenience, the table headers in our abstract model
are in the reverse order, compared to the actual implementation.  Our
abstract model keeps track of the array slots that represent the new
table and its previous version. This information is stored in the
indexes \texttt{new} and \texttt{old}, respectively. The variable
\texttt{cur\_sz} represents the size of the new table. The value of
\texttt{cur\_sz} is doubled each time the hash table needs to grow.

To make our proofs simpler and more efficient, our abstract model also
keeps track of several counters to monitor thread's activities:

\begin{itemize}

\item \texttt{pending}: number of pending additions to the table, this
  is the number of threads at state \pcone.

\item \texttt{migrated\_entries}: number of entries that have been
  copied from the old table into the new one.

\item \texttt{mover\_count}: number of threads that have copied
  entries from the old table to the new one.

\item \texttt{adder\_count}: number of threads that added entries into
  the new table.
  
\item \texttt{freebie\_count}: number of threads that were in state
  \pcone when a new table was allocated. These threads will add
  entries into the new table without copying from the old to the new
  table.

\end{itemize}  

\subsubsection{Transitions}

In this section, we show in detail all the transitions of our abstract
model. Our model makes use of several constant parameters: $N$ is the
number of table headers, $T$ is the occupancy threshold, $M$ is the number of
entries that each threshold needs to migrate before adding a new
entry, and $P$ is the number of threads. For convenience, we assume a
type \texttt{table\_index} which denotes all natural numbers in the
range $[1,\ldots,N]$.

%% N is num of arrays
%% T is the threshold
%% M is the number of entries to migrate
%% S is the current size
%% P is the number of threads

\begin{itemize}

\item \textbf{Init:} 

\begin{minipage}{\linewidth}
\pcode[\small]{
  $state := \pczero$ \\
  $\HT{1,\ldots,N} := \{ num\_to\_entries := 0, num\_to\_migrate := 0, assimilated := false \}$ \\  
  $cur\_sz := 32$ \\
  $old := new := 1$ \\
  $pending := freebie\_count := mover\_count := adder\_count := migrated\_entries := 0$}
\end{minipage}


\item \textbf{Migrate}:  

\begin{minipage}{\linewidth}
\pcode[\small]{
  $state=\pczero \wedge old \neq new \wedge \HT{old}.assimilated \rightarrow$ \\
  \> $num\_to\_migrate := \mathsf{min}(M, \HT{old}.num\_to\_migrate)$ \\
  \> $\HT{new}.num\_entries \incr num\_to\_migrate $ \\
  \> $\HT{old}.num\_entries \decr num\_to\_migrate$ \\
  \> $\HT{old}.assimilated := (\HT{old}.num\_to\_migrate = 0)$ \\
  \> $pending \incr 1$, $mover\_count \incr 1$ \\
  \> $migrated\_entries \incr num\_to\_migrate$\\
  \> $state := \pcone$}
\end{minipage}


\item \textbf{No migrate}: 

\begin{minipage}{\linewidth}
\pcode[\small]{
  $state=\pczero \wedge \negate{(old \neq new \wedge \HT{old}.assimilated)} \rightarrow$ \\
  \> $pending \incr 1$ \\
  \> $state := \pcone$}
\end{minipage}

\item \textbf{Added}: 

\begin{minipage}{\linewidth}
\pcode[\small]{
  $state=\pcone \wedge (\HT{new}.num\_entries < cur\_sz) \rightarrow$ \\
  \> $\HT{new}.num\_entries \incr 1$ \\
  \> $pending \decr 1$ \\
  \> $adder\_count \incr 1$ \\
  \> $state := \pctwo$}
\end{minipage}

\item \textbf{Full}:

\begin{minipage}{\linewidth}
\pcode[\small]{
  $state=\pcone \wedge (\HT{new}.num\_entries \geq cur\_sz) \rightarrow$ \\
  \> $pending \decr 1$ \\
  \> $freebie\_count \decr 1$ \\
  \> $state := \pczero$}
\end{minipage}

\item \textbf{Grow}:

\begin{minipage}{\linewidth}
\pcode[\small]{
  $state=\pctwo \wedge (\HT{new}.num\_entries > (T \times cur\_sz)) \wedge (new < N) \rightarrow$ \\
  \> $cur\_sz := 2 \times cur\_sz$ \\
  \> $old := new$ \\
  \> $new \incr 1$ \\
  \> $\HT{new+1}.num\_entries := 0$ \\
  \> $\HT{new+1}.num\_to\_migrate := 0$ \\
  \> $\HT{new+1}.assimilated := false$ \\
  \> $\HT{new}.num\_to\_migrate := \HT{new}.num\_entries$ \\                
  \> $freebie\_count := pending$ \\
  \> $mover\_count :=  migrated\_entries := adder\_count := 0$ \\
  \> $state := \pczero$}
\end{minipage}

\item \textbf{No Grow}:

\begin{minipage}{\linewidth}
\pcode[\small]{
  $state=\pctwo \wedge (\HT{new}.num\_entries \leq (T \times cur\_sz)) \rightarrow state := \pczero$}
\end{minipage}

\item\textbf{Out of memory}:

\begin{minipage}{\linewidth}
\pcode[\small]{
  $state=\pctwo \wedge (\HT{new}.num\_entries > (T \times cur\_sz)) \wedge (new \geq N) \rightarrow state := \pcthree$  \\
  $state=\pcthree \rightarrow state :=\pcthree$} 
\end{minipage}

\end{itemize}


\subsection{Proving Correctness of a Key Invariant about the Migration Mechanism}

We are interested in proving a key invariant in the implementation of
our hash table. This invariant reflects that migration is always from
the old table to the new one regardless of the number of
copies. Informally, for all indexes $i \in \{1, \ldots, N\}$ such that
$i < old$ then the table header $i$ must be assimilated. This can be stated
more formally as follows:

  \[ \forall 1 \leq i \leq N.~ i < old \Rightarrow \HT{i}.assimilated \]

We assume that there is an arbitrary (but bounded) number of threads,
denoted by $P$, calling the \texttt{lfth\_add} operation in an
asynchronous manner. For convenience, we will refer to $state[i]$ as
the value of the state variable for thread $i$.


Current verification technology can only hope for proving whether a
property is \emph{inductive invariant}. Informally, given a state
machine defined by $Init(x)$, a formula over a set of state variables
$x$ that represents the set of initial states, and $Tr(x,x')$, the
transition relation defining the new values $x'$ from the old values
$x$, we say that a property $Inv(x)$ is inductive invariant if:

\begin{itemize}
\item $Init(x) \Rightarrow Inv(x)$
\item $Inv(x) \wedge Tr(x,x') \Rightarrow Inv(x')$
\end{itemize}

Unfortunately, the key invariant we wish to prove is not inductive
invariant. A common solution to this problem is to find a
\emph{strengthening} of the property we want to prove, $Inv'(x)$, such
that $Inv(x)$ in conjunction with $Inv'(x)$ becomes inductive. That
is,

\begin{itemize}
\item $Init(x) \Rightarrow (Inv(x) \wedge Inv'(x))$
\item $(Inv(x) \wedge Inv'(x)) \wedge Tr(x,x') \Rightarrow (Inv(x') \wedge Inv'(x'))$
\end{itemize}

Complementary to strenghtening, a very powerful verification technique
is $k$-induction~\cite{MouraRS03}, which is a generalization of
induction\footnote{Induction is $k$-induction for $k=1$.}. We say a
property $Inv(x)$ is $k$-inductive if:

\begin{itemize}
\item $Init(x) \Rightarrow Inv(x)$
\item $Inv(x) \wedge Tr(x,x') \wedge \ldots \wedge Inv(x^{k-1}) \wedge Tr(x^{k-1},x^{k}) \Rightarrow Inv(x^{k})$
\end{itemize}


We used the SAL language~\cite{bensalem2000overview,sal2} to encode
the state machine and the key invariant. For the verification engine,
we used Sally~\cite{sally}, a state-of-the-art model checker that
combines k-induction with IC3~\cite{Bradley11}, another powerful
technique to generate inductive invariants.  For that, we translated
the SAL model to the internal format understood by Sally.  It turned
out that Sally was not able to find an inductive strengthening in a
reasonable amount of time.

%% We used Sally~\cite{sally} a state-of-the-art model checker developed
%% by SRI that combines k-induction with IC3~\cite{Bradley11}, another
%% powerful technique to generate inductive invariants. A very appealing
%% feature of Sally is its ability to discover fully automatically
%% non-trivial inductive invariants about state machines that can be used
%% to strengthen the desired property.
%% %
%% Thus, we encoded the state variables and transitions from previous
%% section together with the key invariant we want to prove into Sally
%% format and ran Sally on it.

Since, it is not likely that current verification technology can prove
our key invariant in a fully automatic manner we decided to help the
verification process by manually adding auxiliary lemmas that can make
our property inductive.
%
%% For this, we used SAL~\cite{sal2}, a verification framework
%% developed also by SRI. SAL provides an infinite model checking
%% algorithm based on k-induction which is less powerful than
%% Sally. However, SAL allows to write lemmas in a very expressive
%% language which makes it more amenable for this semi-automatic
%% process.
Prior to presenting all lemmas needed to prove our key
invariant, we define some helper functions:

\begin{minipage}{\linewidth}
\vspace{3mm}
\pcode[\small]{
  $\mathtt{bool}~\mathsf{is\_empty}(hd~:~\mathtt{header}) \equiv
  hd.num\_entries = 0 \wedge hd.num\_to\_migrate = 0 \wedge
  \negate{hd.assimilated}$ \\\\
  $\mathtt{bool}~\mathsf{is\_full}(i~:~\mathtt{table\_index}, k~:~\mathtt{int}) \equiv
  k \geq T \times 2^{(log_{2}~32) + (i-1)}$ \\\\
  $\mathtt{int}~\mathsf{size}(i~:~\mathtt{table\_index}) \equiv
  2^{(log_{2}~32) + (i-1)}$}
\vspace{3mm}
\end{minipage}


\noindent Note that the table indexes ranges from $1$ to $N$. The
definition of \textsf{is\_empty} is straightforward. However, the
definitions of \textsf{is\_full} and \textsf{size} require some
explanation. Since the number of table headers is constant (given by
$N$) instead of using the above definitions we unfold them into a
disjunction of simpler constraints that avoids exponentiation. For
instance, for a given $N=6$ the definition of \textsf{is\_full}
(similarly for \textsf{size}) can be stated as:

\[ (i=1 \wedge k \geq T \times 32) \vee (i=2 \wedge k \geq T \times 64) \vee \ldots \vee (i=6 \wedge k \geq T \times 1024) \]

We are now ready to present all necessary lemmas used to make
inductive our key invariant. All lemmas were constructed by-hand and
represent multiple days of effort. All these lemmas were obtained by
trying to prove each lemma without any strengthening and inspecting a
failed proof for inductiveness which helped us to further strengthen
the lemma.

\begin{lemma}[Relationship between \texttt{old} and \texttt{new}]
  \begin{equation*}
    \begin{array}{l}      
      (new = 1 \wedge old = 1) \vee (new = old+1)
    \end{array}
  \end{equation*}
  \label{old_and_new}
\end{lemma}

\begin{lemma}[All table headers after \texttt{new} must be empty]
  \begin{equation*}
\begin{array}{l}      
  \forall 1 \leq i \leq N.  \wedge i > new \Rightarrow \mathsf{is\_empty}(\HT{i})
\end{array}
\end{equation*}  
\label{empty_beyond_new}  
\end{lemma}  

\begin{lemma}[If a header does not exceed the occupancy threshold then all subsequent headers must be empty]
  \begin{equation*}
\begin{array}{l}        
  \forall 1 \leq i \leq N.  \negate{\mathsf{full}(i,
    \HT{i}.num\_entries)} \wedge (\mathsf{size}(i) =
  \HT{i}.num\_entries) \Rightarrow \\
  \quad \quad  \forall i < j \leq N.  \Rightarrow \mathsf{is\_empty}(\HT{j})
\end{array}
\end{equation*} 
\label{empty_beyond_non_full}
\end{lemma}

\begin{lemma}[Number of pending migrations in a header cannot exceed its number of entries]
  \begin{equation*}
\begin{array}{l}        
  \forall 1 \leq i \leq N. \HT{i}.num\_to\_migrate \leq \HT{i}.num\_entries
\end{array}
\end{equation*}  
\label{num_to_migrate_and_num_entries}
\end{lemma}  

\begin{lemma}[Size of the new table header]
  \begin{equation*}
\begin{array}{l}        
  \mathsf{size}(new) = cur\_sz
\end{array}
\end{equation*}  
\label{table_sizes}
\end{lemma}  

\begin{lemma}[The new table header cannot be assimilated]
  \begin{equation*}
\begin{array}{l}        
  \negate{\HT{new}.assimilated}
\end{array}
\end{equation*}  
\label{new_cannot_be_assimilated}
\end{lemma}  

\begin{lemma}[The new table header cannot have any entry to migrate]
  \begin{equation*}
\begin{array}{l}      
  \HT{new}.num\_to\_migrate = 0
\end{array}
\end{equation*}  
\label{new_cannot_migrate}
\end{lemma}  

\begin{lemma}[The old table header always exceeds the occupancy threshold]
  \begin{equation*}
\begin{array}{l}        
  old \neq new \Rightarrow \mathsf{is\_full}(old, \HT{old}.num\_entries)
\end{array}
  \end{equation*}  
  \label{old_is_full}
\end{lemma}  

\begin{lemma}[If a table header is not full then it cannot be assimilated]
  \begin{equation*}
\begin{array}{l}      
  \forall 1 \leq i \leq N. \negate{\mathsf{is\_full}(i,\HT{i}.num\_entires)} \Rightarrow \negate{\HT{i}.assimilated}
  \end{array}
\end{equation*}
  \label{not_full_cannot_be_assimilated}  
\end{lemma}  

\begin{lemma}[If a table header is assimilated then it cannot have more pending migrations]
  \begin{equation*}
\begin{array}{l}        
  \forall 1 \leq i \leq N. \HT{i}.assimilated \Rightarrow \HT{i}.num\_to\_migrate = 0
\end{array}
\end{equation*}
  \label{assimilated_nothing_to_migrate}  
\end{lemma}  

\begin{lemma}[Relationship between number of pending migrations and assimilated flag in the old headers]
  \begin{equation*}
    \begin{array}{l}
    \forall 1 \leq i \leq N. i \leq old \wedge old < new \Rightarrow  \\
    \quad (\mathsf{is\_full}(i, \HT{i}.num\_entries) \Rightarrow \HT{i}.num\_to\_migrate = 0 \Longleftrightarrow \HT{i}.assimilated) \wedge \\
    \quad (\negate{\mathsf{is\_full}(i, \HT{i}.num\_entries)} \Rightarrow \negate{(\HT{i}.num\_to\_migrate = 0 \wedge \HT{i}.assimilated)})
    \end{array}
  \end{equation*}
  \label{num_to_migrate_and_assimilated_in_old_tables}  
\end{lemma}  

\begin{lemma}[At least one thread must be at the initial state after migration was completed]
  \begin{equation*}
    \begin{array}{l}
      old \neq new \wedge (\HT{old}.num\_entries - \HT{old}.num\_to\_migrate) = 0 \wedge \HT{new}.num\_entries = 0 \Rightarrow \\
      \exists 1 \leq i \leq P.~state[i] = \pczero
    \end{array}
  \end{equation*}
  \label{one_thread_at_init_state_after_migration}
\end{lemma}


\begin{lemma}[Exact bound for pending]
  \begin{equation*}
\begin{array}{l}        
  pending = (\mathsf{if}~state[1]~=\pcone~ \mathsf{then}~1~\mathsf{else}~0) + \ldots + \\
   \quad \quad \quad \quad \quad (\mathsf{if}~state[P]~= \pcone~ \mathsf{then}~1~\mathsf{else}~0)
\end{array}
\end{equation*}  
  \label{pending}
\end{lemma}  

\begin{lemma}[Lower bound for pending]
  \begin{equation*}
\begin{array}{l}        
  pending \geq 0
\end{array}
\end{equation*}  
  \label{pending_lb}
\end{lemma}  


\begin{lemma}[Upper bound for freebie\_count]
  \begin{equation*}
\begin{array}{l}        
  freebie\_count < P
\end{array}
  \end{equation*}  
  \label{did_not_pay_ub}    
\end{lemma}  

\begin{lemma}[Exact bound for mover\_count before having multiple table copies]
  \begin{equation*}
\begin{array}{l}        
  old = new \Rightarrow mover\_count = 0
\end{array}
  \end{equation*} 
  \label{paid_tax0}    
\end{lemma}  

\begin{lemma}[Lower bound for mover\_count]        
  \begin{equation*}
\begin{array}{l}
  mover\_count \geq 0
\end{array}
\end{equation*}
  \label{paid_tax1}  
\end{lemma}


\begin{lemma}[Exact bound for migrated\_entries if multiple table copies]
  \begin{equation*}
\begin{array}{l}        
  old < new \Rightarrow migrated\_entries = \HT{old}.num\_entries - \HT{old}.num\_to\_migrate
\end{array}
\end{equation*}  
  \label{revenue1}    
\end{lemma}

\begin{lemma}[Upper bound for migrated\_entries if multiple table copies]
  \begin{equation*}
\begin{array}{l}        
  old < new \Rightarrow migrated\_entries \leq \HT{old}.num\_entries
\end{array}
\end{equation*}
  \label{revenue2}    
\end{lemma}

\begin{lemma}[Relationship between number of pending migrations in the old header, migrated\_entries and mover\_count]
    \begin{equation*}
\begin{array}{l}      
  old < new \wedge \HT{old}.num\_to\_migrate > 0 \Rightarrow migrated\_entries = T \times mover\_count
\end{array}
    \end{equation*}  
    \label{revenue3}    
\end{lemma}

\begin{lemma}[Lower bound for adder\_count]
    \begin{equation*}
\begin{array}{l}      
  adder\_count \geq 0
\end{array}
    \end{equation*}  
    \label{posted0}  
\end{lemma}

\begin{lemma}[Relationship between assimilated in the old header, pending, mover\_count, adder\_count, and freebie\_count]
  \begin{equation*}
    \begin{array}{l}      
      old < new \wedge \HT{old}.assimilated \Rightarrow adder\_count+ pending = mover\_count + freebie\_count
    \end{array}
  \end{equation*}    
  \label{posted1}
\end{lemma}

\begin{lemma}[Relationship between number of entries in the new header, mover\_count, and adder\_count]
  \begin{equation*}
    \begin{array}{l}      
      old < new \Rightarrow \HT{new}.num\_entries \leq T \times mover\_count + adder\_count
    \end{array}
  \end{equation*}  
  \label{upper_strengthening1}
\end{lemma}

\begin{lemma}[Relationship between number of entries in the new header, number of pending migrations in the old header,  mover\_count, adder\_count]
\begin{equation*}
\begin{array}{l}  
  old < new \wedge \HT{old}.num\_to\_migrate \geq t \Rightarrow
  \\ \HT{new}.num\_entries = T * mover\_count + adder\_count
\end{array}
\end{equation*}
\label{upper_strengthening2}  
\end{lemma}

\begin{lemma}[Exact bound of the number of entries in the new header]
  \begin{equation*}
    \begin{array}{l}  
      old < new \Rightarrow \\
  \HT{new}.num\_entries = (\HT{old}.num\_entries - \HT{old}.num\_to\_migrate) + adder\_count
\end{array}
\end{equation*}
  \label{upper_strengthening3}  
\end{lemma}

\begin{lemma}[Out of memory]
  \begin{equation*}
\begin{array}{l}      
  \exists 1 \leq i \leq P. state[i] = \pcthree \Rightarrow new = N
\end{array}
\end{equation*}  
  \label{out_of_mem}  
\end{lemma}


\begin{lemma}[Upper bound of the number of entries in all headers]
  \begin{equation*}
\begin{array}{l}      
  \forall 1 \leq i < N. \HT{i}.num\_entries \leq (T \times \mathsf{size}(i)) + P
\end{array}
  \end{equation*}
  \label{upper_global}
\end{lemma}

\begin{lemma}[Upper bound of the number of entries in old headers]
  \begin{equation*}
\begin{array}{l}    
  old < new \wedge \negate{\HT{old}.assimilated} \Rightarrow \\
  \HT{new}.num\_entries \leq (\HT{old}.num\_entries - \HT{old}.num\_to\_migrate) + \\
  (\HT{old}.num\_entries - \HT{old}.num\_to\_migrate / T) + (P-1) + pending
\end{array}
\end{equation*}
  \label{upper}
\end{lemma}

\begin{table}[t]
  \begin{center}
    \begin{tabular}{|l|l|}
      \hline
      \textsf{Lemma} & \textsf{Dependencies} \\
      \hline
      \hline     
      \ref{empty_beyond_new} & $\{$\ref{old_and_new}$\}$ \\
      \ref{new_cannot_be_assimilated}  & $\{$\ref{table_sizes}$\}$ \\
      \ref{new_cannot_migrate}  & $\{$\ref{table_sizes}$\}$ \\
      \ref{old_is_full}  & $\{$\ref{table_sizes}$\}$ \\
      \ref{not_full_cannot_be_assimilated}  & $\{$\ref{old_is_full}$\}$ \\
      \ref{assimilated_nothing_to_migrate}  & $\{$\ref{new_cannot_be_assimilated}$\}$ \\
      \ref{num_to_migrate_and_assimilated_in_old_tables} & $\{$\ref{old_and_new},\ref{new_cannot_be_assimilated},\ref{old_is_full}$\}$ \\
      \ref{one_thread_at_init_state_after_migration} & $\{$\ref{old_is_full},\ref{assimilated_nothing_to_migrate}$\}$ \\
      \ref{pending_lb} & $\{$\ref{pending}$\}$ \\
      \ref{did_not_pay_ub} & $\{$\ref{pending}$\}$ \\
      \ref{revenue2} & $\{$\ref{revenue1}$\}$ \\
      \ref{upper_global} & $\{$\ref{old_and_new},\ref{empty_beyond_new},\ref{empty_beyond_non_full},\ref{num_to_migrate_and_num_entries},\ref{table_sizes},\ref{new_cannot_migrate},\ref{old_is_full},\ref{not_full_cannot_be_assimilated},\ref{assimilated_nothing_to_migrate},\ref{one_thread_at_init_state_after_migration},\ref{pending},\ref{pending_lb},\ref{did_not_pay_ub},\ref{paid_tax0},\ref{paid_tax1},\ref{revenue1},\ref{revenue2},\ref{revenue3},\ref{posted0},\ref{posted1},\ref{upper_strengthening1},\ref{upper_strengthening2},\ref{upper_strengthening3},\ref{out_of_mem}$\}$ \\
      
     \ref{upper} & $\{$\ref{old_is_full},\ref{num_to_migrate_and_assimilated_in_old_tables},\ref{pending_lb},\ref{did_not_pay_ub},\ref{revenue1},\ref{revenue2},\ref{revenue3},\ref{posted1},\ref{upper_strengthening1}$\}$  \\
      \hline 
    \end{tabular}
    \caption{Lemma dependencies: the lemma on the left is proven
      assuming the lemmas on the right. Lemmas that do not appear were
      proven without any assumption.}
    \label{table:dependencies}
    \end{center}
\end{table}



\paragraph{Experiments.} To prove the lemmas and the key invariant we used the SAL
k-induction engine. Some lemmas were proven without further
strengthening but some lemmas required the help of other
lemmas. Table~\ref{table:dependencies} shows the dependencies between
lemmas. For the Lemma~\ref{upper_global}, not all the lemmas on the
right were needed but they improved significantly verification time.

\begin{table}[t]
  \begin{center}
    \begin{tabular}{|c|l|l|}
      \hline
      \textsf{\# Threads} & \textsf{K} & \textsf{Time(seconds)} \\
      \hline
      \hline           
      2 & 5  & 0.5\\
      \hline           
      3 & 7  & 1 \\
      \hline           
      4 & 9  & 2\\
      \hline           
      5 & 11 & 6 \\
      \hline           
      6 & 13 & 55 \\
      \hline           
      7 & 15 & 2745 \\      
      \hline 
    \end{tabular}
    \caption{Proving key invariant (and all auxiliary lemmas) with SAL $k$-induction engine}
    \label{sal-results}
    \end{center}
\end{table}


Table~\ref{sal-results} shows the results of running SAL on a 3.5GHz
Intel Xeon processor with 16 cores and 64GB on a Linux machine. We
proved each lemma and the key invariant using the SAL command
\texttt{sal-inf-bmc -i -d $K$ -s yices2}. We proved each lemma in
order based on the dependencies shown in
Table~\ref{table:dependencies}.
%
The first column denoted by \textsf{\# Threads} is the number of
threads, the second column denoted by $\mathsf{K}$ is the induction
depth for the $k$-induction engine, and the third column
(\textsf{Time}) is the accumulated verification times for all lemmas
and the key invariant in seconds. All the experiments assumed that the
maximum number of allocated table headers is three. If this limit is exceeded
the state machine moves to \pcthree\ state.
%
We were able to prove that our invariant holds up to seven
threads. All lemmas were proven using $K=1$, except
Lemma~\ref{upper_global} which needed up to depth $15$ for seven
threads.
%




