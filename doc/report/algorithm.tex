\section{Algorithm}
\label{sec:algo}

We develop a hash table that maps integral keys to integral
values. The implementation supports insertion of new $(key,value)$
pairs, updates of the value assigned to an existing key, removal of a
key, and search. Updates and insertions are implemented by a single
{\em add\/} operation. These operations can be performed concurrently
by multiple threads. In the typical case, none of these operations
require obtaining a lock.

The underlying hash table algorithm we use is linear probing. The
hash-table content is stored in a shared array. Each entry in that
array is a pair that consists of 64-bit integers or pointers. Updates
to this array are performed atomically using compare-and-swap (CAS)
operations. Our algorithm attempts to keep the array less than 70\%
full (this can be configured at compile time). When this threshold is
exceeded, we allocate a fresh array of double the current capacity and
migrate entries from the current array into this fresh array.

\todo{A little sentence here to smooth the transition}.

\subsection{Overview}

Our hash table uses 64-bit keys and values. Table entries must be
aligned on addresses that are multiples of four. These constraints are
imposed by the CAS operations on Intel hardware. We also use the low-order
bit of the keys as a mark.

The hash table consists of a header and a linked list of
dynamically-allocated arrays.  The first element in this list is the
most recent copy of the table. The next element is an older copy and
so forth.  In most cases, threads operate on the most current table
(i.e., the first one in the list). When this table becomes full, we
allocate a freh array of double its size and add it at the front of
the list.  At this point, all threads that operate on the table are
required to migrate entries from the old table to the head of the
list. A key invariant that our algorithm maintains is that migration
is always from the second table in the list to the first table. In
other words, the current table cannot become full before the previous
table has been completely migrated.  We say then that this table is
{\em assimilated\/}.

During the migration process, a thread that operates on the table must
first copy a fixed number $M$ of key/values pairs from the old table
to the new table. The number $M$ is chosen so that a new table cannot
become full before the old table has been assimilated.

To avoid running out of memory, it is desirable to free old copies of
the table that are no longer needed. But, because of thread
scheduling, some threads may keep pointers to old versions of the
tables (i.e., elements deep in the linked list). We can only free old
copies when we are sure that no threads have pointers to them. We use
reference counting for this purpose. Maintaining correct reference
counts introduces race conditions that we could not eliminate with
CAS.  We then introduced a locking mechanism, which is used only in
critical operations that are part of the migration process. Typical
accesses to the table do not require locking.


\subsection{Data Structures}

\begin{figure}
\begin{center}
\begin{clisting}
typedef struct lfht_entry_s {
  volatile uint64_t  key;
  volatile uint64_t  val;
} lfht_entry_t;

typedef struct lfht_hdr_s lfht_hdr_t;

struct lfht_hdr_s {
  atomic_uint_least32_t reference_count;
  atomic_bool assimilated;
  uint64_t sz;
  uint32_t max;
  uint32_t threshold;
  atomic_uint_least32_t count;
  lfht_hdr_t *next;
  lfht_entry_t *table;
};

typedef struct lfht_s {
  uintptr_t hdr:62, state:2;
  pthread_mutex_t lock;
} lfht_t;
\end{clisting}
\end{center}
\caption{Data Structures}
\label{fig:datastructures}
\end{figure}

Figure~\ref{fig:datastructures} shows the main datastructures. A hash
table consists of a {\em descriptor\/} of type \texttt{lfht\_t} that
stores a state, a pointer to the list of table, and a lock. Each
element of the list is a structure of type \texttt{lfht\_hdr\_t}. It
stores an array of entries, counters and sizes, and a pointer to the
next list element.

%% - hdr; the head of
%% the linked list - state; the table state - lock; a mutex

%% In order to be able to update the state and hdr fields of a table
%% atomically, they are stored as bit fields in a single 64 bit field.

%% The linked list consists of one or more table_hdr structures.

%% A table_hdr structure consists of:
%%  - reference_count; the number of threads that can touch this structure
%%  - assimilated; a flag indicating that all the data in the table has been migrated
%%  - count; the number of non-zero keys in the table data
%%  - table; the table data, an array of key value pairs used to accomodate the linear probing hash entries
%% plus some other bookeeping fields that are needed to manage the table_hdr.


%% A handle consists of:
%%  - self; the identity of the thread owning it
%%  - table; the hash table
%%  - table_hdr; the thread's idea of the current table
%%  - last; the oldest table the thread knows about


When a thread creates a handle to a table, it increments the reference
count of the current table, and if the table is expanding, also the
next table in the linked list, which is the handle's last field.  If
the table is not expanding, last is just the current table.

Note that a handle may not be upto date. The list
header may in fact point to a newer table, and not the handle's table\_hdr
field. However on creation we must ensure that it is upto date, and
this is one of the critical sections that we must protect with a lock.

Periodically a thread will ensure that its handle is upto date. During this process
it may move it's last field, and decrement reference counts. Similarly
at periodic intervals the last table in the linked list will be checked to see
if it be reclaimed (i.e. has zero) reference count.


\todo{This is old stuff. Move it to pseudocode?}

So an active table can be in one of three states:

0. the current table is the only table.

1. there is more than one table, the current table is new, and its
predecessor still contains key-value pairs that have not yet been
migrated to the new table.

2. there is more than one table, the current table is new, and its
predecessor table has been assimilated, all key-value pairs have been
migrated to the new table.


So as a first approximation, our hash table consists of a state and a
pointer to the list header, the front of the linked list of tables.
The question then, is "how can we release old tables?" The problem is
that, without additional information, there could always be slow
threads that wake up thinking that the old table is the current active
table.

We solve this problem by using reference counting. A thread's access
to the hash table is via a handle object. The add, remove, and find
operations all must use a handle, rather than the table itself.  The
handle maintains information about the thread's knowledge of the
table. Of particular interest is the thread's current idea of which
is the active table, and what is the oldest table the thread has
an active interest in.



Pseudo Code
-----------

TBD


Critical Sections
-----------------

TDB
