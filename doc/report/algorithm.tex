%% \subsection{Algorithm}
%% \label{sec:algo}

\section{Overview}

Our hash table maps integral keys to integral values. The
implementation supports insertion of new $(key,value)$ pairs, updates
of the value assigned to an existing key, removal of a key, and
search. Updates and insertions are implemented by a single {\em add\/}
operation. These operations can be performed concurrently by multiple
threads. In the typical case, none of these operations require
obtaining a lock.

We use linear probing. The hash-table content is stored in a shared
array. Search for a key start sat some index $i$ derived from the
key's hash value then proceeds incrementally from $i$. Each entry in
the array consists of two 64-bit integers or pointers. Updates to this
array are performed atomically using compare-and-swap (CAS)
operations. Our algorithm attempts to keep the array less than 70\%
full (this can be configured at compile time). When this threshold is
exceeded, we allocate a fresh array of double the current capacity and
migrate entries from the current array into the fresh array.  Removal
of an entry is done by storing a special value called a {\em
  tombstone\/} in the entry's value field. Entries with tombstoned
values are not copied during the migration process.

%% In the rest of this section we give an overview of our approach, the
%% data structures involved, and a tour of a representative sample of the
%% actual code.


%% \subsection{Overview}

Our hash table uses 64-bit keys and values. Table entries must be
aligned on addresses that are multiples of four. These constraints are
imposed by the CAS operations on Intel hardware. We also use the low-order
bit of the keys as a mark.

The hash table consists of a descriptor and a linked list of
dynamically-allocated arrays, called table headers.  The first element
in this list is the most recent copy of the table. The next element is
the previous copy and so forth.  In most cases, threads operate on the
most current table (i.e., the first one in the list). When this table
becomes full, we allocate a fresh array of double its size and add it
at the front of the list.  At this point, all threads that operate on
the table are required to migrate entries from the old table to the
new current table, the head of the list. A key invariant that our
algorithm maintains is that migration is always from the second table
to the first table in the list. In other words, the current table
cannot become full before the previous table has been completely
migrated.  We say then that this older table is {\em assimilated\/}.

During the migration process, a thread that operates on the table must
first copy a fixed number $M$ of key/values pairs from the old table
to the new table. The number $M$ is chosen so that a new table cannot
become full before the old table has been assimilated.

To avoid running out of memory, it is desirable to free old copies of
the table that are no longer needed. But, because of thread
scheduling, some threads may keep pointers to old versions of the
tables (i.e., elements deep in the linked list). We can only free old
copies when we are sure that no threads have pointers to them. We use
reference counting for this purpose. Maintaining correct reference
counts introduces race conditions that we could not eliminate with
CAS.  We then introduced a locking mechanism, which is used only in
critical operations that are part of the migration process. Typical
accesses to the table do not require locking.


\section{Data Structures}

The hash table consists of a {\em descriptor\/} of type
\texttt{lfht\_t} that stores a state, a pointer to the list of table
headers, and a lock.  The pointer to the list and the state are stored
as a single 64~bit integer to allow us to atomically update them.
\begin{center}
\begin{clisting}
typedef struct lfht_s {
  uintptr_t hdr:62, state:2;
  pthread_mutex_t lock;
} lfht_t;
\end{clisting}
\end{center}
Each
element of the list is a structure of type \texttt{lfht\_hdr\_t}. It
stores an array of entries, each of type \texttt{lfht\_entry\_t}, counters and sizes, and a pointer to the
next list element.
\begin{center}
\begin{clisting}
typedef struct lfht_entry_s {
  volatile uint64_t  key;
  volatile uint64_t  val;
} lfht_entry_t;

typedef struct lfht_hdr_s lfht_hdr_t;

struct lfht_hdr_s {
  atomic_uint_least32_t reference_count;
  atomic_bool assimilated;
  uint64_t sz;
  uint32_t max;
  uint32_t threshold;
  atomic_uint_least32_t count;
  lfht_hdr_t *next;
  lfht_entry_t *table;
};
\end{clisting}
\end{center}

The following three routines deconstruct the bit fields of an \texttt{lfht\_t} object.
\begin{center}
\begin{clisting}
static inline lfht_hdr_t *lfht_table_hdr(const lfht_t* ht){
  return (lfht_hdr_t *)(uintptr_t)(ht->hdr << 2);
}

static inline lfht_state_t lfht_state(const lfht_t* ht){
  return ht->state;
}

static inline void lfht_set(lfht_t* ht, lfht_hdr_t * hdr, lfht_state_t state){
  ht->state = state;
  ht->hdr = (uintptr_t)hdr >> 2;
}
\end{clisting}
\end{center}


Even though we can atomically update the \texttt{state} and
\texttt{hdr} fields of a table descriptor, we protect them by a lock.
Critical operations use this lock to prevent concurrent modifications
of the \texttt{state} and \texttt{hdr} fields. The lock is used when a
handle is initialized, when the table grows, and when a thread scans
the list to check whether old assimilated tables can be reclaimed.
%% These are when we first initialize a handle, when we grow the
%% table, and when we see if we can reclaim an old assimilated table.
It is probably possible to implement the latter operation in a
lock-free manner, but we have not found a way to avoid race conditions
in the handle initialization and growing processes that relies only on CAS.


A table can be in one of four states:
\begin{center}
\begin{clisting}
  typedef enum lfht_state {
    INITIAL,
    EXPANDING,
    EXPANDED,
    FINAL
  } lfht_state_t;
\end{clisting}
\end{center}
These correspond to the following situations:
\begin{itemize}
\item INITIAL: The list contains a single, fresh table.
\item EXPANDING: The migration process is in progress: there is more
  than one table, the current table is new and its predecessor
  contains key-value pairs that have not yet been copied into the new
  table.
\item EXPANDED: The migration process has completed: there is more
  than one table, the current table is new and its predecessor has
  been assimilated. All key-value pairs in the predcessor table have
  been copied into the new table.
\item FINAL: The table has been deleted.
\end{itemize}
  
  %% The current table is the only table. 2. There is more than
  %% one table, the current table is new, and its predecessor still
  %% contains key-value pairs that have not yet been migrated to the new
  %% table. 3. There is more than one table, the current table is new,
  %% and its predecessor table has been assimilated, all key-value pairs
  %% have been migrated to the new table. 4. The table has been
  %% reclaimed.

As a first approximation, our hash table consists of a state and a
pointer to the list header, the front of the linked list of tables.
The question then, is ``when can we release old tables?'' Because of
scheduling delays, we cannot ensure that all threads have a
consistent, up-to-date view of the table. Slow threads may wake up and
have pointers to old table headers and mistakenly treat them as the
current table.

We solve this problem by using reference counting.  All hash table
operations must be made using a per-thread {\em handle\/}. The handle
is a structure of type \texttt{lfht\_handle\_t}, and stores the
identifier of the thread that owns it, a pointer to the table, a
pointer to the handle's notion of the current table header, and the
oldest table header that the thread has an interest in:
\begin{center}
\begin{clisting}
typedef struct lfht_handle_s {
  pthread_t owner;
  lfht_t *table;
  lfht_hdr_t *table_hdr;
  lfht_hdr_t *last;  
} lfht_handle_t;
\end{clisting}
\end{center}
%% Of particular interest is the thread's current idea of which
%% is the active table, and what is the oldest table the thread has
%% an active interest in.

When a thread creates a handle to a table, it checks whether the table
is in the expanding state. If so, it stores a pointer to the current
table in \texttt{table\_hdr} and to the previous table in
\texttt{last}, and the thread increments the reference counters of
both table headers. If the table is not expanding, the thread stores a
pointer to the current table in both the \texttt{table\_hdr} and the
\texttt{last} fields, and it increments the current table's reference count.

As other threads access the table, the handle may become out of date:
the \texttt{table\_hdr} may not be the same as the first element in
the hash table's list of headers. However, on creation, we must ensure
that the handle points to the current table and that the reference
counters are incremented atomically. Otherwise, a race condition could
cause the handle to store invalid pointers to table headers that have
been deleted concurrently by another thread. This is one of the
critical sections that protect with a lock. A crucial invariant that
we maintain is that a thread has incremented the reference count of
every table header from the its current \texttt{table\_hdr} to its
\texttt{last} field.

Before any operation on the table, threads attempt to ensure that
their handle is up to date, by updating the \texttt{table\_hdr} and
\texttt{last} pointers and adjusting reference counts. A thread that
sets the reference counter of a table header to zero can scan the list
to determine whether list headers can be deleted. The reference
counters that we maintain are then the number of threads that point to
a particular header through their handle (as opposed to the number of
pointers to each table). Updates to these reference counters are then
relatively infrequent, which is important for performance as they must
be performed atomically.

%% All threads always attempt to 
%% Periodically a thread is up to date. During this process
%% it may move its last field, and decrement reference counts. Similarly
%% at periodic intervals the {\em last} table in the linked list will be checked to see
%% if it be reclaimed (i.e. has zero) reference count.


\section{Algorithms}

We describe the process of initializing a handle, updating a handle,
and using that handle to add an entry to the table. We also describe
the growing of the table and the migration of entries from the
previous table to the current. Creation of the table is
routine and can be gleaned from the source code~\cite{lfht2017}.

\subsection{Handle Initialization and Update}

Initializing a \texttt{lfht\_handle\_t} object is relatively simple.
We set the \texttt{owner} and  \texttt{table} fields to the appropriate objects.
We then block until we have obtained the table's \texttt{lock}. Once we have the lock
we can retrieve the current table header, increment its reference count, and
assign it to the handle's \texttt{table\_hdr} field. If the previous table header
is not assimilated, we presume that the thread maybe required to migrate entries,
and so we increment its reference count as well. Recording the fact that
the \texttt{last} field points to the {\em last}  \texttt{table\_hdr} in the list
whose reference count we have incremented.
We must protect this operation by obtaining the mutex to ensure that the \texttt{actual}
is in fact pointing to the current table header.


\begin{center}
\begin{clisting}
void init_handle(lfht_handle_t *h, lfht_t* table){
  lfht_hdr_t *actual, *next;
  h->owner = pthread_self();
  h->table = table;
  pthread_mutex_lock(&table->lock);
  actual = lfht_table_hdr(table);
  next = actual->next;
  atomic_fetch_add(&actual->reference_count, 1);
  if(next != NULL && !table_is_assimilated(next)){
    atomic_fetch_add(&next->reference_count, 1);
    h->last = next;
  } else {
    h->last = actual;
  }
  h->table_hdr = actual;
  pthread_mutex_unlock(&table->lock);
}
\end{clisting}
\end{center}

While we take care to ensure that upon initialization a thread's
handle is up to date, we cannot maintain such an invariant. However
the handle is designed so that it is particularly easy to determine if
it is up to date or not, using the \texttt{upto\_date} routine.  A
thread's handle is no longer up to date in the situation when the
table has been grown by another thread.  Once a thread determines that
its handle is no longer up to date, it must rectify the problem. 
Because we have no control over thread scheduling, we cannot 
assume that the table has only been grown once. It may have grown several times.
%% In fact it is reasonable to assume that the table has been grown an arbitrary finite
%% number of times.
As a result updating a handle requires not only updating the
\texttt{table\_hdr} and \texttt{last} fields, but also making sure
that the reference counts are accurate. This involves incrementing the
reference counts of all headers that precede \texttt{table\_hdr} in
the list, and decrementing the reference counts of all stale table
header.  This is the reason why we include a \texttt{last} pointer in the handle.
%% It is in this process that we rely on the fact that the
%% handle's \texttt{last} field points to the oldest table whose
%% reference count has been incremented by the handle's owning thread.
Since a handle is local to a thread, there are no risks of no race
conditions when it is updated.

\begin{center}
\begin{clisting}
static bool upto_date(lfht_handle_t *h){
  return lfht_table_hdr(h->table) == h->table_hdr;
}

static void update_handle(lfht_handle_t *h){
  lfht_hdr_t *p;
  bool assimilated;
  lfht_t *table = h->table;
  lfht_hdr_t *actual = lfht_table_hdr(table);
  lfht_hdr_t *current  = h->table_hdr;
  lfht_hdr_t *last = h->last;
  int prior_count = 2;
  if(actual == current){
    return;
  }
  h->table_hdr = actual;
  p = actual;
  while(true){
    atomic_fetch_add(&p->reference_count, 1);
    p = p->next;
    if(p == current){ break; }
  }
  if(current == last){  return; }
  p = current->next;
  while(true){
    assimilated = table_is_assimilated(p);
    assert(assimilated);
    prior_count = atomic_fetch_sub(&p->reference_count, 1);
    if(p == last){ break; }
    p = p->next;
  }
  h->last = current;
  if(prior_count == 1){
    free_table_check(h->table, "update_handle", true);
  }
}
\end{clisting}
\end{center}

\subsection{Addition}

Adding a key-value pair to a table illustrates how we use the
handle. We first compute the hash of the key. We then attempt to
participate in any migrating that might be required. We will describe
the migration process in detail shortly but two properties are
important.  First, the migration scans the old table starting from the
hash of the key we are interesting in. If this key is present in the
old table, the migration will then migrate it to the new opy. This
ensures that the addition will happen after the right key value pair
has been moved to the current table. Second, the migration process
endeavors to ensure that the thread handle is up to date.  Once we
have performed our migration duties, we then proceed to add the
key-value pair to the table by calling function \texttt{\_lfht\_add}.
If this succeeds, and the handle is still up to date, we increment the
table header's count, and check whether the table needs to grow.  If
\texttt{\_lfht\_add} succeeds, but our handle is no longer up to date,
we have added to an old table. Function \texttt{\_lfht\_add} may also fail
if the current table is full.
%% If we did not
%% succeed, then either that table was full, or else the key has already
%% been assimilated (and we are not up to date).
In either case we need to update our handle and try again. If the
table was full, we rely on another thread to grow the table. This is
sound since all threads that successfully added to the table beyond
its threshold will try to grow the table. One will succeed.

After a succesful addition to the most current table, we increment the
table header's entry \texttt{count} and attempt to grow the table if
this count is above the threshold.

\begin{center}
\begin{clisting}
bool lfht_add(lfht_handle_t *h, uint64_t key, uint64_t val){
  bool retval, current;
  lfht_hdr_t *table_hdr;
  uint_least32_t count;
  uint32_t hash = jenkins_hash_ptr((void *)key);
  while(true){
    _migrate_table(h, key, hash);
    retval = _lfht_add(h, key, hash, val);
    current = upto_date(h);
    if( retval && current ){ break; }
    if( ! current ){ update_handle(h); }
  }
  table_hdr = h->table_hdr;
  count = atomic_load(&table_hdr->count);
  if (count >= table_hdr->threshold){
    _grow_table(h->table, table_hdr);
  }
  return retval;
}

\end{clisting}
\end{center}


\subsection{Growing the Table}

A thread calls function \texttt{\_grow\_table} when it detects that
the current table (as seen from the thread's hanlde) is full. Several
threads may attempt to grow the same table concurrently, so we check
whether the thread's handle is up to date. If not, this may indicate
that another thread has successfully grown the table.
%% It is
%% quite likely that there will be contention for this task, so we pass
%% in the thread's current table header so as to be able to detect if the
%% table has already grown.
The actual task of growing the table is relatively simple:  we allocate
a new table header that is twice the size of the current header 
\footnote{Since our main application is a memory allocator, we use
  \texttt{mmap} to allocate table header memory.}  and we set it as
the current header, and we change the table state to
\texttt{EXPANDING}.  We do this under the protection of the lock, not
because of race conditions involved in the process (we
could easily update pointer and state atomically with a CAS), but because other
threads may be in delicate operations, such a initializing a new
handle to the table, or scanning the table header list looking
for tables to reclaim.


%grow_table
\begin{center}
\begin{clisting}
bool _grow_table(lfht_t *ht,  lfht_hdr_t *hdr){
  lfht_hdr_t *ohdr, *nhdr;
  bool retval = false;
  uint32_t omax, nmax;
  pthread_mutex_lock(&ht->lock);
  ohdr = lfht_table_hdr(ht);
  if(hdr == ohdr){
    omax = ohdr->max;
    if (omax < MAX_TABLE_SIZE) {
      nmax = 2 * ohdr->max;
      nhdr  = alloc_lfht_hdr(nmax);
      if (nhdr != NULL){
	nhdr->next = ohdr;
	lfht_set(ht, nhdr, EXPANDING);
	retval = true;
      }
    }
  }
  pthread_mutex_unlock(&ht->lock);
  return retval;
}

\end{clisting}
\end{center}


\subsection{Migration}

The routine \texttt{\_migrate\_table} attempts to move
\texttt{MIGRATIONS\_PER\_ACCESS} key-value pairs from the old table
header to the new table header. The routine is sensitive to the
possibility that the table state may change.  The workhorse of this
process is subroutine \texttt{assimilate} which returns the number of
key-value pairs actually moved.  If this number is less than the
requested amount, we can safely assume that the table has been fully
migrated. In this case we atomically update the \texttt{assimilated}
field of the old table header, and, while holding the lock,
we update the state of the stable from \texttt{EXPANDING} to
\texttt{EXPANDED}.\footnote{Race condition: not requiring that the
  thread updating the table state holds the lock, introduces a nasty
  race condition, where the thread being a bit slow actually toggles
  the state from \texttt{EXPANDING} to \texttt{EXPANDED} for a
  migration that has just commenced, not the previous one that just
  finished.}  As mentioned earlier, when moving key-value pairs from
the old table header to the new table header, we start the
migration at the hash of the key of the API operation that triggered
the call to \texttt{\_migrate\_table}.




%migrate_table
\begin{center}
\begin{clisting}
static int32_t _migrate_table(lfht_handle_t *h, uint64_t key, uint32_t hash){
  int table_state;
  lfht_t *ht;
  lfht_hdr_t *hdr, *ohdr;
  uint32_t moved;
  bool finished, assimilated;
  int32_t migrated = -1;
  
 restart:

  update_handle(h);
  ht = h->table;
  table_state = lfht_state(ht);
  if (table_state == EXPANDING){
    hdr = lfht_table_hdr(ht);
    ohdr = hdr->next;
    if (!upto_date(h) ){ goto restart; }
    if(hdr == h->last){ return migrated; }
    moved = assimilate(h, ohdr, key, hash,  MIGRATIONS_PER_ACCESS);
    finished = moved < MIGRATIONS_PER_ACCESS;
    assimilated = false;
    migrated = moved;
    if (finished  &&  atomic_compare_exchange_strong(&ohdr->assimilated, &assimilated, true)){
      pthread_mutex_lock(&ht->lock);
      if (lfht_state(ht) == EXPANDING  && lfht_table_hdr(ht) == hdr){
	lfht_set(ht, hdr, EXPANDED);
        free_table_check(h->table, "_migrate_table", false);
      }
      pthread_mutex_unlock(&ht->lock);
    }
  }
  return migrated;
}
\end{clisting}
\end{center}

The \texttt{assimilate} routine starts out at the slot corresponding
to the \texttt{hash} and attempts to move the \texttt{key} and its
associated value to the new table if it is present and is not
tombstoned. Along the way it also tries to move \texttt{count - 1}
other non-assimilated non-tombstoned key-value pairs. The actual move
is done by the routine \texttt{\_lfht\_move}, which we will describe
shortly.  After a successful move, a key is marked as assimilated in
the old table (by setting the key's low order bit). Because the mark
happens after the move, several threads may try to move the same
key-value pair but only of them will succeed.  Marking keys before the
move would avoid this issue (as then a single thread would attempt the
move) but this causes a more subtle problem. Moving an entry from the
old to the new table is not atomic and a thread can be interrupted
during the process. Such a thread could mark the last key to be
migrated then go to sleep before completing the copy. In this
situation, all keys are marked in the old table and no other thread
will find anything to migrate. More and more entries may then
be added to the new table before the old table header is marked as
fully assimilated. Marking keys only after the move completes alleviates this problem.

%% but we prefer this to the flawed situation where we
%% first mark the key as assimilated, then move.  The problem with
%% marking the key as assimilated, moving, then updating the counters, is
%% that the last thread to do this (before the table is assimilated) can
%% go to sleep before the increment, and never get to increment the
%% counter before the new table fills up.


%assimilate
\begin{center}
\begin{clisting}
static uint32_t assimilate(lfht_handle_t *h, lfht_hdr_t *from_hdr, uint64_t key, uint32_t hash,  uint32_t count){
  uint32_t retval, mask, j, i;
  lfht_entry_t current, *table;
  uint64_t akey;
  bool success = false  
  retval = 0;
  mask = from_hdr->max - 1;
  table = from_hdr->table;
  if (table_is_assimilated(from_hdr)) {
    return retval;
  }
  j = hash & mask;
  i = j;
  while (true) {
    current = table[i];
    if (current.key == 0 || key_equal(current.key, key)) {
      success = true;
    }
    if (current.key == key ||
	(retval < count && current.key != 0 && !key_is_assimilated(current.key))) {
      if (current.val != TOMBSTONE){
	_lfht_move(h, current.key, jenkins_hash_ptr((void *)current.key), current.val);
      }
      akey = set_assimilated(current.key);
      if(cas_64((volatile uint64_t *)&(table[i].key), current.key, akey)){
	retval ++;
      }
    }
    if (success && retval >= count){
      break;
    }
    i++;
    i &= mask;
    if ( i == j ){
      break;
    }
  }
  return retval;
}
\end{clisting}
\end{center}


\subsection{Adding Entries}

We now describe the two operations that perform linear probing in the
underlying table: \texttt{\_lfht\_move} and \texttt{\_lfht\_add}. The
move operation is the workhorse of \texttt{assimilate}. It is simpler
than \texttt{\_lfht\_add} because it never fails: it either
successfully moves a new key-value pair or does nothing if the key it
is trying to move is already present in the new table. The presence of
this key in the new table indicates that another thread did the move
previously. Function \texttt{\_lfht\_add}, on the other hand, must
handle both the caase of a replacement (i.e., chaging the value
associated with a key that is already present), and the situation
where the key has been marked as assimilated.  This can happen if the
table has grown while this thread was interrupted and asleep, and
other threads have been busily migrating keys in the meantime.

Both the add and move operations take a handle, a key, the hash of the
key, and the desired value. They commence traversing the array at the
position associated with the hash, looking for an empty slot, or a
slot that is already populated by the same key (or an assimilated
variant in the case of \texttt{\_lfht\_add}). If an empty slot is
found, then both operations perform a 128 bit CAS to replace the key
and value in one atomic action.  If the key is already present, then
the \texttt{\_lfht\_move} operation simply returns. The
\texttt{\_lfht\_add} operation performs a 128 bit CAS if the key is
not marked as assimilated, otherwise it fails. Updating the key-value
pair atomically (using a 128-bit CAS) is important to ensure
consistency when the table is growing.\footnote{Race Condition: if we
  did a 64 bit CAS of the key followed by 64 bit CAS of the value,
  then key-value pairs can get lost.  This happens when one thread is
  adding a new key value pair to the table, but is operating on the
  old table. Once it adds the key, another thread that is migrating
  values could look at the key value pair, deduce that it has been
  tombstoned, and mark it as assimilated. In this way the new value
  can be lost.}





%_lfht_move
\begin{center}
\begin{clisting}
static void _lfht_move(lfht_handle_t *h, uint64_t key, uint32_t hash, uint64_t val){
  uint32_t mask, j, i;
  lfht_entry_t*  table, current, desired;
  lfht_t *ht = h->table;
  lfht_hdr_t *hdr = lfht_table_hdr(ht);
  table = hdr->table;
  mask = hdr->max - 1;
  j = hash & mask;
  i = j;
  desired.key = key;
  desired.val = val;
  while (true) {
    current = table[i];
    if (current.key == 0){
      if (cas_128((volatile u128_t *)&(table[i]), *((u128_t *)&current),  *((u128_t *)&desired))){
        atomic_fetch_add(&hdr->count, 1);
        goto exit;
      } else {
        continue;
      }
    }
    if ( current.key ==  key ){
      goto exit;
    }
    i++;
    i &= mask;
    if (i == j) break;
  }
  assert(false);
 exit:
}
\end{clisting}
\end{center}




%_lfht_add
\begin{center}
\begin{clisting}
static bool _lfht_add(lfht_handle_t *h, uint64_t key, uint32_t hash, uint64_t val){
  uint32_t mask, j, i;
  lfht_entry_t*  table, current, desired;
  lfht_t *ht = h->table;
  lfht_hdr_t *hdr = lfht_table_hdr(ht);
  bool retval = false;

  table = hdr->table;
  mask = hdr->max - 1;
  j = hash & mask;
  i = j;
  desired.key = key;
  desired.val = val;
  while (true) {
    current = table[i];
    if (current.key == 0){
      if (cas_128((volatile u128_t *)&(table[i]), *((u128_t *)&current),  *((u128_t *)&desired))){
        atomic_fetch_add(&hdr->count, 1);
	retval = true;
        goto exit;
      } else {
        continue;
      }
    }
    if ( key_equal(current.key, key) ){
      if( key_is_assimilated(current.key) ){
        retval = false;
        goto exit;
      } else {
	if (cas_128((volatile u128_t *)&(table[i]), *((u128_t *)&current),  *((u128_t *)&desired))){
          retval = true;
          goto exit;
        } else {
          continue;
        }
      }
    }
    i++;
    i &= mask;
    if (i == j) break;
  }
 exit:
  return retval;
}
\end{clisting}
\end{center}
