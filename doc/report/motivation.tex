\section{Motivation}
\label{sec:intro}

Our interest in lock-free hash tables arose naturally from the needs
of a metadataless version of the glibc memory allocator.  The glibc
allocator attempts to minimize contention in multithreaded
applications by maintaining a pool of independent heaps called \emph{arenas}.

In a call to free, when chunks are preceeded by metadata, one can use
simple offset calculations, and pointer chasing tricks to go from a
chunk to it's arena. When metadata has been separated from the chunk
this is no longer possible, and as a result we use lock free hash
tables in order to determine the arena associated with the arbitrary
chunk.


Implementing a fixed size lock-free linear probing hash table is a
relatively simple task, there are many examples to be found online.
Adding the additional requirement that it be dynamic, adds some
interesting complications. Also requiring that old copies of the
tables be released back to the operating system turns the task into a
research project.

