\section{Motivation}
\label{sec:intro}

Our interest in lock-free hash tables arose naturally from the needs
of a metadataless version of the glibc memory allocator.  The glibc
allocator attempts to minimize contention in multithreaded
applications by maintaining a pool of independent heaps called \emph{arenas}.

In a call to free, when chunks are preceeded by metadata, one can use
simple offset calculations, and pointer-chasing tricks to go from a
chunk to its arena. When metadata is separated from the chunk,
this is no longer possible, and as a result we use lock-free hash
tables in order to determine the arena associated with an arbitrary
chunk.

Implementing a fixed size lock-free linear probing hash table is a
relatively simple task. Many examples can be found online. An
important aspect that is often not address is to allow the table to
grow as more entries are added. This is key to ensure efficiency and
low hash-collision rates. Growing the table leads to maintaining
several verisons of the table: when the table grows, one must create a
new larger, initially empty table and transfer the content from the
current table to the new table. This creates a new difficulty: when
and how can we reclaim the memory used by the old copy, in a lock-free
fashion?


