\section{Windows}

Microsoft's Windows memory allocator has been the subject of regular
attention from the hacking community. In response, Microsoft has
developed particular security solutions for Internet Explorer and
legacy systems going back to Windows XP. As part of their ongoing
security efforts, Microsoft has also been improving the general resistance
of the default \texttt{malloc} implementation that ships with Windows.
Here, we focus particularly on the design in Windows 8.

\subsection{Browser-specific memory allocators}

Circa 2014, Internet Explorer introduced two new
defenses~\cite{yason2014,simonz2014}. 
\begin{description}
\item [Memory protector] Newly freed objects are zeroed out but are
  not immediately made available for reuse. Instead, a mark-and-sweep
  style system detects whether there are live references from the
  current call stack to any of these heap objects after they've been
  freed. This is effective at defeating ``use after free'' attacks.
\item [Isolated heap] This exposes multiple memory arenas to the
  user level, similar to Apple's \texttt{libmalloc} (see
  Section~\ref{sec:libmalloc}). Operations that could
  potentially cross arenas, like a \texttt{realloc}, will fail. This
  allows Internet Explorer to isolate potentially attackable objects (e.g., aspects
  of the HTML DOM) from system objects. This could help defeat some
  classes of heap overflow attacks.
\end{description}

In Windows 10, Microsoft rolled out ``MemGC'', which provides additional
memory protections for its Edge and Internet Explorer 10
browsers~\cite{yason2015}. MemGC provides a concurrent mark-and-sweep
garbage collector, which originates with Microsoft's JavaScript
implementation, but which also serves the rest of the browser runtime
system.

\subsection{Enhanced Mitigation Experience Toolkit}

Also in 2014, Microsoft released the fifth version of its Enhanced
Mitigation Experience Toolkit (EMET5)~\cite{metcalf2014}. EMET5 can be
applied to versions of Windows as old as Windows XP, and has a variety
of features, such as being able to prevent an application like Microsoft Word
from loading an Adobe Acrobat plugin. To hinder heap attacks,
EMET5 adds:
\begin{description}
\item [Heap spraying] Preallocates blocks in regions that known
  heap-spray attacks like to use. (Trivial for new attacks to work
  around.)
\item [Null page] The page at the bottom of memory is pre-allocated
  to defeat attacks meant to trick the memory allocator into giving
  a reference to the zero / null address.
\item [ASLR + DEP] In addition to defending against a variety of other
  attacks, EMET adds some randomness to where \texttt{malloc}'s pages
  will be located.
\end{description}

\subsection{Windows 8}
The Windows 8 heap memory allocator is sufficiently complicated that
any discussion here will necessarily be superficial.

The basic design of Microsoft's heap allocator is very similar to
other arena-based allocators, with a front-end ``low fragmentation
heap'' (LFH) that manages large numbers of small allocations, and a
back-end that allocates memory from the operating system and feeds it
to the LFHs, as necessary.

Where Windows Vista and earlier versions arranged free blocks as a
doubly-linked list, Windows 8 now uses a bitmap, placing critical
metadata out of the reach of attackers. Similarly, where metadata is
still present, it is generally encoded (i.e., XORed with a constant),
to make it hard for attackers to do anything useful.

The most comprehensive explanation and analysis of the Windows 8 heap
allocator can be found in Valasek and Mandt~\cite{valasek2012}. They
describe the structures and algorithms in great detail.
%
Johnson and
Miller~\cite{johnson-miller2012} discuss ongoing mitigations within Microsoft's heap allocator.
With specific reference to heap
allocation attacks, a Microsoft
technical report~\cite{microsoft2013} summarizes the changes in
Windows 8:

\begin{description}
\item [Heap integrity checks] \texttt{malloc} and \texttt{free}
  perform checks on the metadata and terminate the computation if
  corruption is detected.
\item [Fatal exceptions] In older versions, when \texttt{malloc} or
  \texttt{free} had some sort of error, they would swallow it,
  possibly returning \texttt{null}, and thus giving attackers multiple
  attempts at any given attack. All errors are now immediately fatal.
\item [Freeing metadata] Hawkes~\cite{hawkes2008} shows how a
  use-after-free attack, where the attacker can overwrite a pointer to
  the heap metadata, could then allow an attacker to attack the heap
  metadata structure itself. Windows 8 has specific logic in it to
  detect this attack.
\item [Encoded metadata] A function pointer in each heap header is
  XORed with a constant to make it unusable by an attacker. That
  constant is now relocated to where it's (hopefully) not easily read
  or written by an attacker. Encoding is also used to protect metadata
  for the ``low fragmentation heap''.
\item [Guard pages] All large allocations ($>1$MB on a 64-bit machine)
  go directly to the VM system, which creates a trailing guard
  page. Smaller allocations come from ``heap segments'' which also now
  have trailing guard pages. In some cases, these heap segments are
  broken up, nondeterministically, with additional guard pages.
\item [Allocation ordering] Allocations are also done in a a
  nondeterministic order, making it harder for an attacker to do Heap
  Feng Shui.
\end{description}

Seeley~\cite{seeley2012} observes that, while allocation order is
randomized, the higher-level block allocations made from the operating
system won't be. He estimates a lower-bound $50\%$ chance of
successfully performing a heap attack on Windows 8, while pointing out
how much easier it would be on Windows 7 and
earlier. Liu~\cite{liu2013} makes similar observations.
