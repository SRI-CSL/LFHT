%% \subsection{Patching our glibc into a standard Linux distribution}

While it is desirable to add our improved \texttt{malloc} implementation
to a Linux distribution's glibc, this will not happen all at
once. Some applications will slow down. Others might have mysterious
crashes, perhaps due to their own bugs being sensitive to the behavior
of \texttt{malloc}, or perhaps due to subtle bugs in \texttt{sri-glibc}.
This suggests that systems administrators need a
methodology for rolling out our modified glibc under two
different circumstances: {\em default opt-in with exceptions} and {\em
default opt-out with exceptions}. In other words, we need a mechanism
that can select the new glibc as the system-default, or not, and
then we need a mechanism to identify specific applications and say
that these specific applications should get something other than the
system-default, whatever that might be.

%% \subsection{Changing the system default}

It is certainly possible to replace the shared object file
for \texttt{glibc}, but this does not allow for exceptions. It is
necessary to allow the original glibc and our improved glibc to be
installed side-by-side, with a way to instruct Linux to, by default,
prefer the new glibc over the original.

%% \subsubsection{Relevant environment variables}

The following Linux environment variables impact how shared libraries are loaded:

\begin{description}

\item[LD\_PRELOAD]
Specifies one or more shared libraries, separated by spaces or colons,
to be loaded by a process. These are loaded in advance of anything
specified by the program itself. If SRI's glibc were listed in
\texttt{LD\_PRELOAD}, then it would be loaded instead of the system's
glibc.

\item[LD\_LIBRARY\_PATH]
Specifies a directory to be used in addition to, and in advance of,
any of the existing places that Linux looks for shared libraries. If
SRI's glibc were stored in its own directory and that directory
were in \texttt{LD\_LIBRARY\_PATH}, then it would be loaded instead of
the system's glibc.

\item[/etc/ld.so.preload]
This file is roughly equivalent to \texttt{LD\_PRELOAD}, specifying
shared libraries to be loaded by a process in advance.

\item[/etc/ld.so.conf]
This file specifies the standard locations from which shared libraries
are found. It's only consulted after all the preceding features have
failed to find the library. It typically includes a directive to
``include'' a series of other files in \texttt{/etc/ld.so.conf.d}, each
of which specifies one or more directories in which to find shared
libraries.

\end{description}

Setuid programs operate in ``secure mode,'' which changes their
behavior with respect to the environment variables above. In
particular, \texttt{LD\_PRELOAD} and \texttt{LD\_LIBRARY\_PATH} are ignored
when running a setuid program. (Otherwise, a user could compile
arbitrary code into a replacement for glibc, run any setuid
program, and thus arrange for it to run the user-supplied code as
root.) The file \texttt{/etc/ld.so.preload}, however, is always used,
even for setuid programs. This makes it or \texttt{/etc/ld.so.conf} the
preferable solution for doing a system-wide change for a system
default.\footnote{There are also some variables that the linker looks
for inside these paths, some of which are still available in secure
mode. This formed the basis for a root exploit discovered by Tavis
Ormandy in
2010. \url{http://seclists.org/fulldisclosure/2010/Oct/257}}


%% \subsubsection{Relevant ELF headers}

Modern Linux executable files are structured according to the ELF
(``Executable and Linkable Format'') standard. A full discussion of
the ELF standard is beyond the scope of this document, but it is
important to note that an ELF binary has fields that tell the linker
which shared libraries it wants. Those fields are:

\begin{description}

\item[DT\_RPATH]
A list of directories to look for a library. This list is consulted
{\em before} \texttt{LD\_LIBRARY\_PATH}. This header is considered
``deprecated'' but it still works.
\item[DT\_RUNPATH]
A list of directories to look for a library. This list is consulted
{\em after} \texttt{LD\_LIBRARY\_PATH}. If the \texttt{runpath} is set, then
the \texttt{rpath} is ignored.

\item[DT\_NEEDED]
A list of the shared libraries used by this application. These are
commonly named {\em without} fully qualified paths, and then resolved
with the various search paths specified above. A library specified in
the \texttt{needed} variable may also have a fully qualified path, and
will thus be resolved to that path.

\end{description}

To make an exception to the system-wide policy, as specified in the
\texttt{LD\_} environment variables or \texttt{/etc/ld.so.*} disk files, the
obvious place to start is to edit a given program's \texttt{rpath} or
\texttt{runpath} variables, allowing that program to specify its own
directories in which to find shared libraries. Alternately, the \texttt{needed}
variable can be edited to point specifically to the SRI glibc.

We surveyed a variety of different tools and techniques for
manipulating the \texttt{rpath}, \texttt{runpath}, and \texttt{needed}
variables in an ELF binary. The best solution seems to be
PatchELF\footnote{\url{https://nixos.org/patchelf.html}}. Notably, it
has flags like \texttt{--add-needed}, \texttt{--replace-needed} which
provide a safe way to manipulate existing binaries. By specifying the
SRI glibc as an absolute path with the \texttt{--replace-needed}
flag, the original glibc can be replaced with the new one.

The \texttt{readelf} tool is part of most standard Linux
distributions. It's useful for looking at an existing binary to see
what it's doing. For example, running \texttt{readelf -d} will print the
``dynamic'' section, which includes all the \texttt{needed} variables,
thus listing all the shared libraries that will be requested when this
program is loaded. (Those libraries might have their own downstream
dependencies. These aren't shown.) This is useful, when looking at an
existing binary, to verify whether it's been patched beforehand with
\texttt{patchelf}.

The \texttt{ldd} tool is part of most standard Linux distributions. You
simply run \texttt{ldd} with the name of a binary and it lists each
shared library that will be requested as well as the disk file to
which it will be mapped. This process will recursively follow the
dependencies from each library as well. This is also useful, when
looking at an existing binary, to verify whether it's been patched
properly with \texttt{patchelf}.
