\section{Apple (OS X and iOS)}
Security researchers have paid attention to
Apple operating systems,~\cite{reverse}
but its memory allocator has received less attention.
Some discussion appears in Miller and
Zovi~\cite{mac-hackers-handbook-2009,ios-hackers-handbook-2012}
in 2009 and 2012.  The earliest public discussion seems to be from
Nemo~\cite{nemo2005} in 2005.

As of 2016, Apple uses two different \texttt{malloc} implementations:
\texttt{libmalloc} and \texttt{bmalloc}. \texttt{libmalloc} appears to date
from Apple's acquisition of NeXT, while \texttt{bmalloc} is a brand new
system, used in Apple's Safari browser. Both are distributed as open
source on Apple's web site\cite{libmalloc, bmalloc}. \texttt{bmalloc} is also distributed
as part of Apple's open source Webkit~\cite{webkit} (i.e., the open
source project from which Apple's Safari browser is derived).


\subsection{LibMalloc}
\label{sec:libmalloc}

\texttt{libmalloc} is implemented in C and exports a superset of the
usual \texttt{malloc} routines. Notably, it has a concept of {\em zones},
allowing a program to allocate fresh memory from any zone. Under the
hood, these zones are essentially the same as the {\em arenas} used in
the glibc memory allocators. With Apple
explicitly exposing these zones, a multi-threaded program can
explicitly manage multiple zones to avoid locking contention when
memory is allocated.

Zones also support {\em batch allocation}, facilitating
efficient  allocation of large numbers of objects of the same size.
Apple's developer pages promote these and other \texttt{libmalloc}-specific
features.\footnote{\url{https://developer.apple.com/library/mac/documentation/Performance/Conceptual/ManagingMemory/Articles/MemoryAlloc.html}}
Apple
also supports zone-specific allocation in its Cocoa Objective-C
libraries\footnote{\url{https://developer.apple.com/library/ios/documentation/General/Conceptual/CocoaEncyclopedia/ObjectAllocation/ObjectAllocation.html}}
and apparently zone allocation is also used inside the newer Swift
programming
environment.\footnote{\url{http://www.russbishop.net/swift-how-did-i-do-horrible-things}}

The downside of \texttt{libmalloc}'s approach, however, is that
alternative \texttt{malloc} implementations do not implement the various
zone-related methods and are not drop-in compatible with programs or
libraries that expect to have the extended \texttt{libmalloc} features.

\texttt{libmalloc} gives explicit credit to the Hoard allocator as an
influence on its design. This analysis focuses on management of the ``tiny''
regions, used for small object sizes.

Tiny objects are allocated from {\em regions}.
Each region is 1MB in
size and holds 64520 16-byte {\em blocks}.
A region also has a {\em region trailer} which contains the
metadata for the region.
The trailer includes a bitmask to track the free/busy
state of each tiny block. When a block is allocated, it has no
per-block header or trailer. Free blocks are maintained using a
doubly-linked list.

Apple's design, in the presence of a use-after-free situation,
potentially allows attackers to have access to the free-list
pointers. Furthermore, if an attacker can overflow the last block in a
region, the attacker will then have access to that region's metadata.

%% \begin{figure}
%% % \lstset{language="pascal",showstringspaces=false,basicstyle=\ttfamily}
%% % \begin{lstlisting}[frame=single]
%% \begin{description}
%% \item[MallocLogFile $<f>$] to create/append messages to file $<f>$ instead of stderr
%% \item[MallocGuardEdges] to add 2 guard pages for each large block
%% \item[MallocDoNotProtectPrelude] to disable protection (when previous flag set)
%% \item[MallocDoNotProtectPostlude] to disable protection (when previous flag set)
%% \item[MallocStackLogging] to record all stacks.  Tools like leaks can then be applied
%% \item[MallocStackLoggingNoCompact] to record all stacks.  Needed for
%%   {\sf malloc\_history}
%% \item[MallocStackLoggingDirectory] to set location of stack logs,
%%   which can grow large; default is {\sf /tmp}
%% \item[MallocScribble] to detect writing on free blocks and missing initializers: 0x55 is written upon free and 0xaa is written on allocation
%% \item[MallocCheckHeapStart $<n>$] to start checking the heap after $<n>$ operations
%% \item[MallocCheckHeapEach $<s>$] to repeat the checking of the heap after $<s>$ operations
%% \item[MallocCheckHeapSleep $<t>$] to sleep $<t>$ seconds on heap corruption
%% \item[MallocCheckHeapAbort $<b>$] to abort on heap corruption if $<b>$ is non-zero
%% \item[MallocCorruptionAbort] to abort on malloc errors, but not on out of memory for 32-bit processes. MallocCorruptionAbort is always set on 64-bit processes
%% \item[MallocErrorAbort] to abort on any malloc error, including out of memory
%% % \end{lstlisting}
%% \end{description}
%% \caption{Environment variables supported by {\sf libmalloc}.\label{fig:libmalloc-env}}
%% \end{figure}

\texttt{libmalloc} maintains a checksum in its heap metadata, added
sometime between 2009 and 2012, that defeats some heap metadata
overwrite attacks. A checksum is stored in the high four bits of
each free-list pointer, based on a randomly-generated
cookie~\cite{ios-hackers-handbook-2012}.

\texttt{libmalloc} also has a variety of flags that radically change how it
operates, along with a related {\em Guard Malloc}
library\footnote{\url{https://developer.apple.com/library/ios/documentation/Performance/Conceptual/ManagingMemory/Articles/MallocDebug.html}}. These help developers in
finding memory errors in their programs. They have a serious
impact upon performance. They could potentially be
repurposed for security if their performance could be
improved.
%% Figure~\ref{fig:libmalloc-env}
%% provides a list of the environment variables that control the security
%% checking.
Of note, \texttt{MallocGuardEdges} adds guard pages around large blocks, which will
cause memory overruns to abort. Also, several
environment variables instruct \texttt{libmalloc} to perform consistency
checks on its heap. If corruption is detected, the process will abort.

\subsection{BMalloc}
Before 2015, the Webkit browser used Google's \texttt{tcmalloc}, then
Apple abruptly replaced it with \texttt{bmalloc}. 

The \texttt{bmalloc} implementation does not
presently have the standard \texttt{malloc} API. Instead, a family of
similarly-named calls (``mbmalloc'', ``mbcalloc'', ``mbfree'', etc.)
are provided, along with an internal switch that can route these calls
to the ``real'' \texttt{malloc} implementation or handle them
internally. A program linked against \texttt{bmalloc} can then
be switched over to a third-party \texttt{malloc} implementation.
\texttt{bmalloc} does not expose anything
analogous to the ``zone'' features of \texttt{libmalloc}.

\texttt{bmalloc} has a unique design. It could be best described as ``lazy'', in that
when a block is freed, a pointer to it is placed in a fixed-size
buffer. In the common case, allocation and free operations are
constant time, with a very low cost.

\texttt{bmalloc} supports multiple {\em heaps}. A heap has {\em small
  pages} which in turn have {\em small lines}.
\texttt{bmalloc} tracks the free space within these small  lines.
Then, for each of a
variety of {\em size classes}, \texttt{bmalloc} maintains ``bump
allocators'' which are backed by those small lines. Objects are
allocated starting at the beginning of a line and continuing
sequentially to the end. 

Each heap has an asynchronous ``scavenger'' which processes its free lists,
looking for memory it can reuse. If the scavenger gets behind, the
normal allocation and memory freeing routines can also do the same
work.

Unlike \texttt{libmalloc}, memory objects in \texttt{bmalloc} never hold
metadata when they're freed. Instead of an inline free list, there is a
counter in the header for each line of memory to track how much is in
use versus how much is free. If the counter indicates a line of memory is
entirely unused, it can be recycled for future allocations.

The metadata resides in headers at the front of each heap. The
asynchronous nature of the free-list scavenger makes it harder for an attacker
to predict the state of the heap at any given time, which is otherwise
essential to Heap Feng Shui attacks.

\texttt{bmalloc} has no explicit features to detect 
use-after-free or double-free operations which might be exploited
as part of an attack, but it has clearly been engineered to make
some attacks more difficult.

